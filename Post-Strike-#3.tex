\documentclass{article}

\title{Recording Techniques Post-Strike \#3}
\author{Alexei Ozerov}
\date{}

\begin{document}

\maketitle

\section{Class Introduction}

\begin{itemize}
\item Guitars.
\item Look at ``My Recording Rig'' assignment.
\item Refresh yourself on how to make spreadsheets.
	\begin{itemize}
	\item Price, Qty, Subtotal, Tax, Total
	\item Use equations.
	\end{itemize}
\item Brian is going to be teaching the class next week.
\item There will be a few investigations today.
\end{itemize}

\medskip

Reflection will be due tomorrow.

\medskip

Read ``Physical and Perceived Elements of Sound''.

\medskip

\section{Investigation \#1}

Acoustic Guitar Mic Placements

\section{Investigation \#4: E. Guitar Placement}

\begin{itemize}

\item 1. Edge of the Cone - Darker Sound 

\item 2. Between Edge and Center of the cone - Thinner Sound 

\item 3. Center of the cone - VERY Brittle

\item 4. On edge and angled - Compared to 2, this is slightly brighter, with a more defined low 

\item 5. Royer - Condensor, more proximity effect, smooth, I like it

\end{itemize}

Audiofile Calculator.

\medskip

Measured two wavelengths - set SM57s down at 1/2 Wavelength, 1 wavelength, and two wavelengths. The fundamental note was an ``E2''.

\begin{itemize}
\item Close - Playing ... add further and further away bits 
\item 1/2 Wavelength - Grainy, Flanger-ish, pretty shitty sound, kills the fundamental because of phase 
\item 1 Wavelength - Fundamental of Tonic is Reinforced (Sounds like a doubled guitar), sounds sick with the 1 Wavelength added 
\item 2 Wavelengths - Sounds iffy, a bit too far away, and sounds strange, time delay makes it sound off.
\end{itemize}

\section{Investigation \#6: Jack Richardson Techniques}

Three Microphones On The Cone:

\begin{itemize}
\item Shure SM57 - one side of the cone at an angle, close
\item Sennheiser 421 - other side of the cone 
\item Neumann TLM 170 - center
\end{itemize}

Lots of polarity flipping and messing with relative level. U87 sounded squirrely, 427 sounded clear and powerful. On their own, they were all pretty okay, but it was fantastic when combined. Lots of different things you could do with the tone. 57 and 421 sometimes get the same signal, so you can mess around with polarity. 




\end{document}
