\documentclass{article}

\title{Mixing Demo}
\author{Alexei Ozerov}
\date{December, 18th/2017}

\begin{document}

\maketitle
\tableofcontents

\section{Introduction}

Brian Martin will demo the mixing (in the box) of three student tunes.

\subsection{The Approach}

The Truth (in 2017):

\medskip

\begin{itemize}
\item Melody, Harmony, Rhythm - Europe (Art)
\item Rhythm, Melody, Harmony - Africa (Practical)
\item Melody + Rhythm - Current
\item The mix \textbf{must have melody + rhythm}. Don't screw up the groove, don't mess up the vocal.
\item If the mix doesn't make you move, if you have to think, it is not working.
\item Engage people viscerally - then add harmony, etc.
\end{itemize}

\medskip

Art is \textbf{expression}.

\medskip

Get your \textbf{ego} out of the mix.

\medskip

Everything you do has a price, and minimalism is something to do.

\medskip

Try pulling the faders down \& removing plugins, and building up the mix from the beginning, and re-engage with the mix.

\section{Song \#1: Chris}

\begin{itemize}
\item Begins by setting up the balance of the drums
\item DI is a waste of time
\item Avoid using ``solo'' - the interrelationships of sounds is what makes the mix
\item Mixes relatively quietly 
\item Overheads, Kick, Snare (all you need for the drums)
\item Reverb Delay panned right if mono original is left 
\item Avoid stereo everywhere - phase issues,etc.
\item 
\item 
\item
\item 
\item 
\item 
\item 
\item 
\item
\end{itemize}













\end{document}
