\documentclass{article}

\author{Alexei Ozerov}
\title{A Review of Recording Resources - Mastering Audio}
\date{December 2nd, 2017}

\begin{document}

\maketitle
\tableofcontents
\begin{center}
\noindent\rule{12cm}{0.4pt}
\textit{``Mastering is the art of compromise. It is the art of knowing what is sonically possible, and then making informed decisions about about what is most important for the music (Katz, 2002)''.}
\end{center}

\vspace{0.6cm}

\section{Introduction}

``Mastering Audio - the art and the science'', by Bob Katz (2002), is a book covering the author's experiences and practices as a mastering engineer. This resource is written for both those aspiring to become mastering engineers, and those who just want to know what they can do to their mix to prepare it for such an engineer, and what will happen to their tracks.

\medskip

This resource drew me to it because of my interest in the technical side of working in a studio. After seeing my own mix session, I was incredibly interested in finding out exactly what each dip in the EQ did, and became curious as to what happens next.
As it so happens, I will continue with that line of curiousity, and I will be focusing my review on Chapter 8 of this book; ``Equalization Techniques''. 

\pagebreak

\section{Reviewing Chapter 8}

\subsection{A Brief Summary}

The chapter kicks off with Mr. Katz speaking about how the different practices associated with EQ are both similar and different within mixing and mastering. The author stresses the importance of considering how each alteration in the EQ will change not only the intended instrument, but other elements of the mix.
It is incredibly important to maintain various levels of attention when EQing, as the intended aesthetic of a song or program should always come first, and little ``defects'' may be there for a reason. Scarier yet, you fixing a ``defect'', can ruin another integral part of the mix. Therefore, it is extremely important to not let your ears get tired, and to listen to the various elements of the mix and overall program you are given. From each individual instrument, to the interrelationships of the instruments, to the interrelationship of the mixes of each song.

\medskip

The chapter continues with discussing the different types of EQs (parametric and shelving), how Q works, some basic techniques for removing unwanted resonance, and how to calibrate the Q. There are examples and pictures to go along with each explaination, and the chapter covers; Shelving Filters, Baxandall Curves, and Highpass and Lowpass filters. From there, the author continues onto detailed explainations of how to approach EQing a song, while reminding the reader that each case will be different, and one must use one's own musical, aesthetic, and technical judgement.

\medskip

Mr. Katz continues, again, speaking in more detail about ``Ying and Yang'', and how altering one frequency with the EQ can completely change another part of the song. One has to strive for a balance, and know what effect each alteration to the frequencies will have. The author also stops again to say that the best mix may require no mastering - that mastering is just giving the music exactly what it needs, and not over-indulging. The next few sections of the chapter go into listening for the effect your EQ has, and how this is not a place where you want to make instant decisions. Live with a careful EQ choice, and after time, you will decide.

\medskip

Finally, to wrap up this section, the author speaks about Linear-Phase Equalizers, and Dynamic Equalization, which flew a little over my head. I will have to devote some more time and effort to research this topic, and I got swept up in the technical language. I will give the index a visit as well. This is how the chapter ends.

\pagebreak

\subsection{The Review}

I enjoyed reading chapter 8 in detail. I skimmed through some other chapters as well, and intend to revisit them in more detail after submitting this assignment. I found that the earlier parts of the chapter were incredibly interesting - there was detail about the mechanics of these EQs, about things to always consider, which got me thinking about my own level of awareness, or lack thereof. This sparked an interest in me to see if I can learn more about getting better at hearing the subtle changes one small change in the EQ can bring about. I also enjoyed reading about when and for what the author would consider using each EQ. This helped put the job into perspective for me. However, I will say this - I did not enjoy this resource as much as I anticipated because of how quickly it moved from topic to topic. This could be because I started at chapter 8, but at the same timeI wish there was more information and examples about each type of EQ, and its uses.

\medskip

This could be me realizing that I prefer technical manuals and in-person instruction over a textbook (which differs from a manual as a textbook has much more to cover than one specific topic ... at least in this case). While this resource was great for getting me to think about the things brought up in the chapter and want to seek out more information, it did not cover everything I wanted to know. I guess it may also be aimed towards those who have mastering and mixing experience already. I left with the feeling that I would still be a little lost if I were dropped into a session and told to do my first master.

\section{Recommendation}

In short, I would recommend this book to anyone interested in the subject matter. While chapter 8 was not as in-depth as I had hoped, I have a feeling that the rest of the book may fill those voids. I also believe that while this is not a resource I would keep looking back to while mastering, it would be something which shapes the way I approach a mastering project. The ideas of careful application of EQ, and listening to what a piece of music (or pieces of music) really need is something that has, and will continue to stick with me.

\medskip

I am excited to see what others think of this book, and I am personally excited to finish reading it from the top.

\section{References}

Katz, Bob. Mastering Audio the Art and the Science. 1st ed., Focal Press, 2002. 

\end{document}
